\section{Introducción}

\borrador{
  Párrafo de introducción a lo que vendrá después.\\
  Contar sobre el surgimiento de nuevas herramientas para enseñar a programar y sus características (componentes visuales, no pensadas para el ámbito industrial, requieren trabajo por parte de los docentes para crear ejercicios más completos, intentan ser atractivas). Se podrían nombrar algunos ejemplos: Scratch, Pilas Bloques, Wollok, Gobstones y Mumuki.
}

\subsection{Gobstones, una nueva forma de aprender a programar}
\borrador{
  \begin{itemize}
    \item qué es Gobstones (lenguaje + secuencia didáctica)
    \item qué herramientas existen hoy en día
    \item dónde se usa (unq, belgrano, sarmiento, alcal, ameghino)
    \item existe un equipo de desarrollo, liderado por el Lic. Ary Battista, que se puso a disposición del desarrollo de este trabajo
  \end{itemize}
}

\subsection{Mumuki, educación libre de la programación}
\borrador{
  \begin{itemize}
    \item qué es el proyecto Mumuki
    \item por qué herramientas está compuesto actualmente (todas o sólo las que son de interés para este trabajo?)
    \item dónde se usa (utn, unq, undav, alcal)
    \item existe un equipo de desarrollo, liderado por el Ing. Franco Bulgarelli, que se puso a disposición del desarrollo de este trabajo
  \end{itemize}
}
