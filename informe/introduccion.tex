\newcommand{\caracteristica}[1]{\item \textbf{#1:}}

\section{Introducción}
La informática ha tenido sus inicios entre matemáticos (como Alan Turing\footnote{Turing} y John von Neumann\footnote{VonNeumann}) y, algunas décadas después, con la invención del circuito integrado, los electrónicos. Debido a las limitaciones tecnológicas de aquellas primeras computadoras, es razonable que la primera percepción de programación estuviera fuertemente acoplada a la máquina, ya que resultaba imposible pensar una cosa sin la otra. Como ejemplo de esto, basta mencionar la conocida anécdota del \textit{bug}\footnote{Bug}, que persiste hoy en día como metáfora asociada a una falla dentro de un programa.
\\\\
\borrador{
Mini mini reseña de cómo se fue pasando del bajo al alto nivel, y cómo esto era necesario porque el hardware no daba para más.
}
\\\\
Esta manera de percibir a la programación puede verse reflejada en muchos cursos iniciales, donde se hace énfasis en cuestiones que tienen más que ver con el \textit{hardware} subyacente que con el \textit{software} que se pretende construir: se favorece a la performance del programa por sobre otros atributos como expresividad, abstracción, declaratividad. Si bien este enfoque es válido en ciertos contextos, en muchos otros (y sobre todo en un primer acercamiento a la programación) no sólo que no aportan sino que muchas veces lo único que se logra es ahuyentar a los estudiantes, transmitiendo que la programación es un arte oscuro e incomprensible, exclusivo para unos pocos.
\\\\
En contraposición a este enfoque, surge una nueva forma de enseñar a programar que pretende ser más inclusiva y más didáctica, haciendo uso intensivo de herramientas tecnológicas para asistir al docente en su rol de educador. Algunas herramientas que han ido surgiendo en los últimos años son: Logo, Alice, Scratch, y más recientemente Pilas Bloques, Wollok, Gobstones y Mumuki. \borrador{citar o explicar qué son todas esas cosas que nombré}.
\\\\
Si bien cada una de esas herramientas es diferente, podemos identificar ciertas características compartidas:
\begin{itemize}
  \caracteristica{componentes visuales}
  \caracteristica{pensadas para el ámbito educativo}
  \caracteristica{requieren un trabajo fuerte por parte del docente}
  \caracteristica{esconden aspectos netamente tecnológicos que no hacen al pensamiento computacional}  
\end{itemize}

En este trabajo nos concentraremos en Gobstones y Mumuki, y en cómo hacer para que se puedan conectar.

\subsection{Gobstones, una nueva forma de aprender a programar}
\borrador{
  \begin{itemize}
    \item qué es Gobstones (lenguaje + secuencia didáctica)
    \item qué herramientas existen hoy en día
    \item dónde se usa (unq, belgrano, sarmiento, alcal, ameghino)
    \item existe un equipo de desarrollo, liderado por el Lic. Ary Battista, que se puso a disposición del desarrollo de este trabajo
  \end{itemize}
}

\subsection{Mumuki, educación libre de la programación}
\borrador{
  \begin{itemize}
    \item qué es el proyecto Mumuki
    \item por qué herramientas está compuesto actualmente (todas o sólo las que son de interés para este trabajo?)
    \item dónde se usa (utn, unq, undav, alcal)
    \item existe un equipo de desarrollo, liderado por el Ing. Franco Bulgarelli, que se puso a disposición del desarrollo de este trabajo
  \end{itemize}
}
