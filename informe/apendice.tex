%!TEX root = main.tex

\renewcommand{\appendixname}{Apéndice}
\renewcommand{\appendixtocname}{Apéndice}
\renewcommand{\appendixpagename}{Apéndice}

\begin{appendices}

\label{app:Specs}
\section{Ejemplos de pruebas en Stones Spec}

\begin{listing}
  \centering

  \begin{minted}{yaml}
  check_head_position: true
  examples:
   - initial_board: |
       GBB/1.0
       size 2 2
       head 0 0
     final_board: |
       GBB/1.0
       cell 1 1 Rojo 1
       size 2 2
       head 1 1
  \end{minted}

  \caption{Representación YAML de un test de \emph{programa}, que chequea que haya una bolita roja en la celda \codigo{(1, 1)} y que el cabezal se encuentre allí.}
  \label{lst:ProgramSpec}
\end{listing}

\begin{listing}
  \centering

  \begin{minted}{yaml}
  subject: PonerN
  examples:
   - arguments:
      - 2
      - Azul
     initial_board: |
       GBB/1.0
       size 2 2
       head 0 0
     final_board: |
       GBB/1.0
       size 2 2
       cell 0 0 Azul 2
       head 0 0
  \end{minted}

  \caption{Representación YAML de un test de \emph{procedimiento}, que chequea que \codigo{PonerN} pone 2 bolitas azules cuando es llamado con los argumentos \codigo{(2, Azul)}.}
  \label{lst:ProcedureSpec}
\end{listing}

\begin{listing}
  \centering

  \begin{minted}{yaml}
  subject: nroBolitasTotal
  examples:
   - initial_board: |
       GBB/1.0
       size 2 2
       cell 0 0 Rojo 1 Azul 2 Negro 2 Verde 5
       head 0 0
     return: 10
  \end{minted}

  \caption{Representación YAML de un test de \emph{función}, que chequea que \codigo{nroBolitasTotal} devuelve 10 en una celda con 1 bolita roja, 2 azules, 2 negras y 5 verdes.}
  \label{lst:FunctionSpec}
\end{listing}

\end{appendices}
