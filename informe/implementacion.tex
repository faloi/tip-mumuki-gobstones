%!TEX root = main.tex

\section{Implementación}

\subsection{Cómo atacar los requisitos}
\borrador{
  \begin{enumerate}
    \item PyGobstones provee una forma para ejecutar programas por consola
    \item no había nada
    \item PyGobstones provee una forma para obtener el AST de un programa por consola, pero tenía algunos problemas:
      \begin{itemize}
        \item no había forma de pedirle que no ejecutara el programa, por lo tanto no se podían parsear programas que producieran \textit{BOOM}
        \item todas las funciones y procedimientos tenían que estar definidos, por lo tanto no se podían parsear soluciones que dependieran de código adicional (sobre el que no interesa correr ningún tipo de validación porque lo provee el docente)
        \item la representación del AST no es standard (aún no resuelto)
      \end{itemize}
    \item había una representación pero le faltaba trabajo, no se comprendía por completo y no respetaba la identidad visual de Gobstones
  \end{enumerate}
}

\subsection{Extensiones a Mumuki}
\borrador{
  \begin{itemize}
    \item poder mostrar el resultado de una ejecución, para no perder el componente visual de Gobstones. Hasta el momento no existía ya que en otros lenguajes no interesaba mostrar el resultado
    \item soportar HTML como output de un runner
  \end{itemize}
}

\subsection{Tecnología elegida}
\borrador{
  \begin{itemize}
    \item la interacción con PyGobstones se realizaría por consola, por lo cual la elección del lenguaje no influía en nada
    \item ya existía mumukit, una biblioteca para Ruby (se la llama gema) que resolvía todo lo que tenía que ver con el manejo HTTP
  \end{itemize}
}

\subsection{Stones Spec}

\subsection{Mumuki Gobstones Server}
\subsubsection{Integración con Stones Spec}
\subsubsection{Chequeo de expectativas}
