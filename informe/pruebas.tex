%!TEX root = main.tex

\section{Pruebas}
\subsection{Testing automatizado del código}
\borrador{
  Porcentaje de cobertura, tipos de tests que existen.
}

\subsection{Utilización de las herramientas en un colegio secundario}
Si bien las pruebas automatizadas brindan cierto nivel de confianza sobre la calidad del software, nada nos dicen sobre su utilidad. Es por ello que en la propuesta del presente trabajo se proponía la creación de 3 guías de ejercicios que permitieran poner en funcionamiento la herramienta con estudiantes.

Al momento de escribir este informe me desempeño como docente de un curso de programación destinado a estudiantes del quinto año del Instituto Sagrado Corazón Al.Cal, bajo la orientación Tecnicatura en Programación Informática. Gran parte del interés por la temática elegida surge de mi inquietud como educador de mejorar constantemente mis prácticas, utilizando la tecnología como medio para poder realizar un seguimiento más personalizado de mis estudiantes. Es por ello que las guías creadas (introducción, procedimientos y repetición simple) tuvieron a mis estudiantes como destinatarios, sirviendo en algunos casos como repaso de temas ya vistos y en otros como primer acercamiento a conceptos nuevos.

Los resultados fueron muy positivos. Al ofrecerles un entorno más controlado y feedback automatizado sobre los errores, muchos de ellos pudieron terminar de comprender los temas propuestos. Gracias al análisis de las soluciones enviadas, tuve la oportunidad de ver estados intermedios que dan cuenta de cómo cada estudiante aprende de sus errores, algo que es muy difícil lograr sin una herramienta de este estilo (sería necesario poder sentarse varias horas al lado de un estudiante viendo cómo elabora su solución, procurando no influenciar en el proceso).
