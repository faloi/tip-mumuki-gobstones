%!TEX root = main.tex

\section{Requerimientos}

\subsection{Mumuki}

\subsubsection{Arquitectura}
\borrador{
  \begin{itemize}
    \item \textit{plataforma:} donde el estudiante sube sus soluciones y ve el feedback
    \item \textit{runners:} servicios web encargados de probar los programas que escribe el estudiante
    \item diagrama de secuencia de la arquitectura:
      \begin{itemize}
        \item alumno envia solución
        \item plataforma envia solución, tests, expectativas y código adicional en una request al runner
        \item el runner compila, corre los tests, chequea las expectativas y devuelve el resultado
        \item la plataforma renderiza los resultados
      \end{itemize}
  \end{itemize}
}

\subsubsection{Requisitos para poder implementar un runner}
\borrador{
  \begin{enumerate}
    \item ejecutar un programa Gobstones
    \item probar que el resultado (tablero final o valor de retorno) coincide con lo esperado
    \item una representación del AST para poder evaluar las expectativas
    \item una representación HTML de los tableros para poder mostrarle al estudiante
  \end{enumerate}
}

\subsection{Gobstones}

\subsubsection{Arquitectura}
\borrador{
  PyGobstones consta de un compilador/intérprete y de una GUI. En este trabajo no utilizaremos la GUI, ya que está construida para una aplicación Desktop y no puede ser reutilizada dentro un sitio web.
}
