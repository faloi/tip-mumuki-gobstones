%!TEX root = main.tex

\section{Requerimientos}
En esta sección veremos cómo funcionan las herramientas involucradas y cuáles son los puntos de extensión que proveen.

\subsection{Mumuki}

\subsubsection{Arquitectura}
Antes de empezar a hablar de la arquitectura, es necesario entender cómo se estructura el contenido dentro de la plataforma, ya que esto impacta en cada uno de los componentes.

El ejercicio es la mínima unidad de contenido: se trata de una descripción de un problema y una forma de evaluarlo. Desde el punto de vista del estudiante, consta de un título, una descripción del problema, una serie de ayudas adicionales y un corolario (que se muestra tras resolver el problema correctamente). Desde el punto de vista técnico, se estructura de la siguiente manera:

\begin{itemize}
  \caracteristica{Pruebas} evalúan que la solución resuelva el problema de forma correcta, es decir, que llegue al resultado esperado.

  Por ejemplo, si tenemos una función \codigo{dobleDe(numero)} que recibe un número y devuelve su doble, una prueba posible sería que \codigo{dobleDe(2)} da como resultado \codigo{4}.

  \caracteristica{Expectativas} evalúan que la solución resuelva el problema de forma adecuada, es decir, que se utilicen las herramientas correctas.

  Supongamos que ahora queremos hacer la función \codigo{cuadrupleDe(numero)}, que tome un número y devuelve su cuadruple, y queremos que el estudiante la implemente en términos de \codigo{dobleDe} (ya que multiplicar por 4 es lo mismo que multiplicar 2 veces por 2). En ese caso nuestro objetivo sería \codigo{cuadrupleDe(numero) \textit{debe usar} dobleDe(numero)}.

  \caracteristica{Código adicional} herramientas que el estudiante puede utilizar al escribir su solución. Pueden ser fruto de ejercicios anteriores o simplemente partes del problema que el docente quiera facilitar.

  Retomando el ejemplo anterior, la función \codigo{dobleDe(numero)} podría ser provista por el docente, ya que se supone que el estudiante pudo razonarla en el ejercicio anterior y sería tedioso que tuviera que volver a escribirla.
\end{itemize}

De los componentes mencionados en la introducción, sólo nos interesan el Atheneum y los \runner s ya que como vemos en la Figura \ref{fig:FlujoSubmission} son los que intervienen en el proceso que va desde el envío de la solución del estudiante hasta la obtención del resultado.

\begin{figure}[FlujoSubmission]
  \centering

  \begin{sequencediagram}
    \newthread{estudiante}{Estudiante}{}
    \newinst{atheneum}{Atheneum}{}
    \newinst{runner}{Runner}{}

    \begin{call}{estudiante}{Validar(solucion)}{atheneum}{resultado}
      \begin{call}{atheneum}{Validar(solucion, pruebas, expectativas, codigo\_adicional)}{runner}{resultado}
      \end{call}
    \end{call}
  \end{sequencediagram}

  \caption{Comunicación entre el estudiante, el Atheneum y un Runner.}
  \label{fig:FlujoSubmission}
\end{figure}

\begin{itemize}
  \item{El estudiante envia su solución, lo cual produce un POST HTTP hacia el Atheneum.}
  \item{El Atheneum envia otro POST HTTP al \runner\ correspondiente, agregando las pruebas, expectativas y código adicional del ejercicio.}
  \item{El \runner\ responde al pedido HTTP con el resultado de evaluar la solución, que puede ser: \codigo{errored} si el programa no compila, \codigo{passed\_with\_warnings} si el programa es correcto (pasa las pruebas) pero no utiliza las herramientas adecuadas o \codigo{passed} si es correcto y cumple con las expectativas.}
\end{itemize}

\subsubsection{Requisitos para poder implementar un runner}
\borrador{
  \begin{enumerate}
    \item ejecutar un programa Gobstones
    \item probar que el resultado (tablero final o valor de retorno) coincide con lo esperado
    \item una representación del AST para poder evaluar las expectativas
    \item una representación HTML de los tableros para poder mostrarle al estudiante
  \end{enumerate}
}

\subsection{Gobstones}

\subsubsection{Arquitectura}
\borrador{
  PyGobstones consta de un compilador/intérprete y de una GUI. En este trabajo no utilizaremos la GUI, ya que está construida para una aplicación Desktop y no puede ser reutilizada dentro un sitio web.
}
