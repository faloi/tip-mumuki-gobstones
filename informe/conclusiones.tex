%!TEX root = main.tex

\section{Conclusiones}
Este trabajo abordó la temática de construir un \runner\ de Gobstones para la plataforma Mumuki. Para llevarlo a cabo, creé un framework de testing automatizado y un servidor web que posibilitara la comunicación con Mumuki Atheneum, la herramienta mediante la cual los estudiantes resuelven ejercicios. En el trayecto también hice contribuciones tanto en \pyGob\ como en varios componentes de Mumuki, incorporando nuevas funcionalidades necesarias para llevar a cabo la integración e incluso resolviendo algunos bugs.

Si bien todavía faltan funcionalidades para considerar que la integración con Gobstones está completa, el producto construido por el presente trabajo resulta funcional y deja planteadas varias líneas de trabajo para continuar su desarrollo. Pueden fácilmente escribirse nuevas guías de ejercicios y extender el uso de la plataforma a cualquier asignatura que utilice Gobstones como soporte para la enseñanza de la programación.

En cuanto a lo académico, considero que pude aplicar gran parte de los conceptos que se trabajan a lo largo de la carrera, tanto desde lo metodólogico como desde los saberes técnico-específicos. En particular hice intensivo uso de todo lo relacionado al paradigma de objetos, aplicando los conceptos aprendidos durante las cursadas dentro de dos lenguajes completamente nuevos para mí (Ruby y Python); demostrando así que las ideas priman por sobre las tecnologías y que aprender un lenguaje nuevo es una tarea prácticamente trivial para un Técnico Universitario en Programación Informática. Aprendí también cuestiones básicas sobre teoría de lenguajes, necesarias para poder interpretar y analizar un AST.

Por último, me siento muy contento de haber podido contribuir a la Comunidad de Programación Informática de nuestra casa con esta herramienta, aportando desde mi lugar para poder seguir mejorando la calidad de la enseñanza de nuestras carreras.
